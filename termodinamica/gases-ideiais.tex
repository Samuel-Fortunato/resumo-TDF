\section{Gases ideais}
Nesta secção indroduz-se o modelo do gas ideal

\subsection{Equação de estado do gás ideal}
Como visto na secção anterior, quando a pressão é baixa relativamente à pressão critica, e/ou quando a temperatura é elevada em relação à temperatura crítica, o fator de compressibilidade é aproximadamente 1. Nestas situações, assume-se $Z=1$, ou seja:
\begin{equation*}
    pv = RT
\end{equation*}
conhecida como a \textbf{equação de estado do gás ideal}.

Existem formas alternativas da mesma equação, obtidas com as relações $v=\frac{V}{m}$, $v=\frac{\overline{v}}{M}$ e $R=\frac{\overline{R}}{M}$:
\begin{align*}
    pV &= mRT \\
    p\overline{v} &= \overline{R}T \\
    pV &= n\overline{R}T
\end{align*}



\subsection{Modelo do gás ideal}

O modelo do gás ideal é definido por 3 equações, que podem ser usadas com um certo grau de erro, uma vez que gases reais se aproximam deste modelo.
\begin{align}
    pv &= RT \label{eq:ideal-gas-1} \\
    u &= u(T) \label{eq:ideal-gas-2} \\
    h &= h(T) = u(T) + RT \label{eq:ideal-gas-3}
\end{align}



\subsection{Energia interna, entalpia, e calores específicos de gases ideiais}

Num gás ideal, como a energia interna $u$ específica depende apenas da temperatura $T$, o calor específico $c_v$ também depende apenas de $T$.
\begin{equation*}
    c_v(T) = \frac{du}{dT}
\end{equation*}

De igual forma, como a entalpia especifica $h$ depende apenas de $T$, então o calor específico $c_P$ também é uma função apenas da temperatura.
\begin{equation*}
    c_p(T) = \frac{dh}{dT}
\end{equation*}

Diferenciando a equação \eqref{eq:ideal-gas-3} em relação à temperatura, obtém-se:
\begin{equation*}
    \frac{dh}{dT} = \frac{du}{dT} + R
\end{equation*}
de onde se obtém a relação:
\begin{equation*}
    c_p(T) = c_v(T) + R
\end{equation*}

Sabe se também que $c_p > c_v$, logo, a razão de calores específicos, $k=\frac{c_p(T)}{c_v(T)}$, é funcão apenas da temperatura e $k>1$.
