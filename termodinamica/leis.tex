\section{Resumo: Leis da termodinâmica}

\subsection{Lei zero da termodinâmica}
\begin{center}
    \noindent\fbox{
        \parbox{0.8\textwidth}{             \centering\textit{
                Dois sistemas em equilíbrio térmico com um terceiro estão em equilíbrio térmico entre si.
            }
        }
    }
\end{center}



\subsection{1ª Lei da Termodinâmica}
\begin{center}
    \noindent\fbox{
        \parbox{0.8\textwidth}{             \centering\textit{
                A variação de \textbf{energia total} de um sistema fechado é igual à diferenca entre a energia transderida para o sistema sobre a forma de calor e o trabalho realizado pelo sistema sobre a vizinhança.
            }
        }
    }
\end{center}

Expressa pela equação:
\begin{equation*}
    \Delta U = Q - W
\end{equation*}








\subsection{2ª Lei da Termodinâmica}
Existem 3 formas de enunciar a \textbf{2ª lei da termodinâmica}:
\begin{itemize}
    \item Enunciado de Clausius
    \item Enunciado de Kelvin-Planck
    \item Enunciado da entropia
\end{itemize}
Os enunciados de Clausius e de Kelvin-Planck são as formulações mais comuns da 2ª lei, mas o enunciado da entropia é o mais útil, e aplicável a uma grande variedade de contextos. Este último enunciado irá ser estudado no próximo capítulo.

\subsubsection{Enunciado de Clausius}
\begin{center}
    \noindent\fbox{
        \parbox{0.8\textwidth}{             \centering\textit{
                É impossível para qualquer sistema operar de forma que o \textbf{único} resultado seja uma transferência de calor de um corpo mais frio para um mais quente.
            }
        }
    }
\end{center}

O enunciado de Clausius não impossibilita a transferencia de calor de um corpo mais frio para um mais quente, mas tal transferência implica outros efeitos no sistema, na vizinhança, ou em ambos.



\subsubsection{Enunciado de Kelvin-Planck}
\begin{center}
    \noindent\fbox{
        \parbox{0.8\textwidth}{             \centering\textit{
                É impossível para qualquer sistema operar num ciclo termodinâmico e fornecer energia por trabalho à sua vizinhança enquanto recebe energia por transfeência de calor de \textbf{um único reservatório} de temperatura.
            }
        }
    }
\end{center}
O enunciado de Kelvin-Planck não impossibilita um sistema de realizar trabalho a partir de calor transferido de um único reservatório, mas implica que esse sistema não realiza um ciclo.

O enunciado de Kelvin-Planck pode ser expressado analiticamente pela equação:
\begin{equation*}
    W_{ciclo} \leq 0\quad\text{(com um único reservatório)}
\end{equation*}



\subsubsection{Enunciado da entropia}
A entropia é uma propriedade extensiva de um sistema, tal como a massa e a energia, mas ao contrário destas, a entropia pode ser gerada quando existem \textit{irreverssibilidades} no sistema.
\begin{center}
    \noindent\fbox{
        \parbox{0.8\textwidth}{             \centering\textit{
                É impossível para qualquer sistema operar de forma a que entropia seja destruida.
            }
        }
    }
\end{center}