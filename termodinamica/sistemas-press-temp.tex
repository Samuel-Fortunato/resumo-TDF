\section{Sistemas e grandezas de estado. Pressão e temperatura.}

\subsection{Sistemas Termodinâmicos}

Um \textbf{sistema termodinâmico} é uma porção do universo
constituída por matéria e/ou radiação e separada da sua
\textbf{vizinhança} por paredes reais ou imaginárias, nomeadas
de \textbf{fronteira}.

\subsubsection{Classificação de Sistemas}

Os sistemas termodinâmicos podem ser classificados como:

\begin{itemize}
  \item \textbf{Fechados -} Permitem a troca de energia com a vizinhança, mas não de matéria.
  
  \item \textbf{Abertos -} Permitem a troca de energia e massa com a vizinhança.
  
  \item \textbf{Isolados -} Não permitem a troca de energia nem massa com a vizinhança.
\end{itemize}

Sistemas fechados são comummente chamados de \textbf{massas de controlo} e sistemas abertos de \textbf{volumes de controlo}



\subsection{Propriedades}

Uma \textbf{propriedade} é uma caraterística de um sistema termodinâmico à qual se pode atribuir valores quantitativos em cada instante sem conhecer os processos pelos quais o sistema passou.

Isto é, uma \textbf{propriedade} é uma caraterística mensurável de um sistema, que é independente do que aconteceu antes no sistema.

Ao conjunto dos valores das propriedades de um sistema chama-se \textbf{estado do sistema}.

\subsubsection{Propriedades intensivas e extensivas}

As propriedades de um sistema dividem-se em:

\begin{itemize}
    \item \textbf{Propriedades intensivas - } propriedades cuja magnitude é independente do tamanho do sistema.
    
    \textit{e.g.:} Temperatura, $T$; Densidade, $\rho$; Temperatura de fusão, $T_m$,
    
    \item \textbf{Propriedade extensivas - } propriedades cuja magnitude é aditiva para subsistemas.

    \textit{e.g.:} Massa, $m$; Volume, $V$; Rigidez da mola, $k$,
    
\end{itemize}



\subsection{Equilíbrio}

Diz-se que um sistema está num \textbf{estado de equilibrio termodinamico} quando as condições da vizinhança são fixas e o sistema evoluiu para um estado que não varia com o tempo e que é independente dos estados anteriores.

Este estado só se alterará se a vizinhança do sistema for sujeita a alterações.



\subsection{Sistemas compressiveis simples}

Quando analisamos \textbf{sistemas compressíveis simples} (sistemas macroscópicamente homogeneos, isotropicos e não carregados) aplica se o seguinte \textbf{postulado de estado}:

\begin{quotation}
    O estado de um \textbf{sistema compressível simples} é completamente especificado por duas propriedades independentes e pela massa de cada substãncia que o constitui.
\end{quotation}

Um sistema compressível simples é geralmente caracterizado pela pressão, $p$, temperatura, $T$, volume, $V$, e pela sua composição (\textit{e.g.} massa dos elementos). Logo, pelo \textit{postulado de estado} uma das propriedades do sistema é determinadas pelas outras duas atraves de uma \textbf{equação de estado}:
\begin{equation*}
    X_3 = f(X_1, X_2)
\end{equation*}
onde $X_i$ representa uma variável de estado.



\subsection{Processos}

Um processo é a mudança que um sistema sofre de um estado de equilibrio para outro. Se qualquer propriedade do sistema muda diz se que ocorre um processo.

\paragraph{Processo Adiabático}
Processo que ocorre sem transferência de energia por calor.

\paragraph{Processo Diabático}
O oposto de adiabático, processo em que ocorre tranferência de energia por calor.

\paragraph{Processo Isobárico}
Processo realizado a pressão constante.

\paragraph{Processo Isotérmico}
Processo ao longo do qual não há mudança de temperatura.

\paragraph{Processo Isocórico}
Processo que ocorre sem alteração de volume do sistema.

\paragraph{Processo Isopícnico}
Processo a densidade constante

\paragraph{Processo Politrópico}
Processo que obedece a relação: $PV^n = \text{constante} $


\subsection{Volume Específico}

O \textbf{Volume Específico ($ v $)} é definido como o inverso da densidade:
\begin{equation*}
    v = \frac{1}{\rho}
\end{equation*}
sendo a densidade dada pela expressão já conhecida: $\rho = \frac{m}{V}$



\subsection{Pressão}

A \textbf{Pressão ($ P $ ou $ p $)} é a força exercida perpendicularmente por unidade de área de uma superfície.

\begin{equation*}
    p = \lim_{A \to A'}\left(\frac{F_{normal}}{A}\right)
\end{equation*}
sendo $A'$ a menor área para a qual existe um valor definido da expressão.\footnote{$A'$ é suficientemente grande para efeitos de partículas individuais serem desprezáveis, mas suficientemente pequena para ser considerada um ponto}

\subsubsection{Medições de Pressão}

\begin{itemize}
    \item \textbf{Pressão Absoluta -} obtida por aplicação direta da fórmula acima. É medida em referencia a um vácuo absoluto, onde a pressão é 0.
    
    \item \textbf{Pressão Atmosférica -} pressão (absoluta) do ar na atmosfera terrestre, que varia principalmente com a altitude, entre outros fatores. Tem o valor ao nível do mar tabelado de: \qty{101,325}{Pa}, \qty{1013.25}{mbar}, \qty{760}{mmHg}, \qty{29.9212}{inHg} ou \qty{14.696}{psi}.
    
    \item \textbf{Pressão Manométrica -} pressão medida relativamente à pressão atmosférica local. Tem o valor zero quando a pressão do fluido é igual à pressão atmosférica.
    \[ p_{gauge} = p_{abs} - p_{atm} \]
    
    \item \textbf{Pressão de Vácuo -} módulo de uma pressão manométrica negativa. Usada quando a pressao a a ser medida é inferior à atmosférica, isto é, existe um vácuo.
\end{itemize}

\subsubsection{Equilibrio hidrostático}

Quando um fluído se encontra num campo gravítico $\vec{g}$, a pressão cresce na direção do campo. A equação obtém-se aplicando a 2ª Lei de Newton a um volume infinitesimal.
\begin{equation*} \label{eq:equilibrio-hidrostatico}
    d\vec{F}_{press} = -\left(\frac{\partial p}{\partial x} \hat{i} + \frac{\partial p}{\partial y} \hat{j} + \frac{\partial p}{\partial z} \hat{k} \right) dx dy dz
\end{equation*}
obtendo-se, de forma mais geral:
\begin{equation*}
\label{equation:hidrostatic-equilb}
    \vec{\nabla} p = \rho \vec{g}
\end{equation*}



\subsection{Temperatura}
\subsubsection{Escalas de temperatura}

Para a definição de uma escala de temperatura e necessário a escolha de dois estados de referência, chamados \textbf{pontos fixos}. O estado escolhido é normalmente o ponto triplo da àgua.

\paragraph{Escala Kelvin}
Baseia-se no ponto triplo da àgua, atribuindo o valor de \qty{273.16}{K} à temperatura nesse estado. O valor de \qty{0}{K} representa o zero absoluto, e temperaturas inferiores a esta não estão definidas.

\paragraph{Celcius}
Baseia-se no \textbf{ponto de fusão do gelo} e no \textbf{ponto de ebulição da água} atribuindo-lhes, respetivamente, os valores de \qty{0}{\degreeCelsius} e \qty{100}{\degreeCelsius}

\paragraph{Fahrenheit e Rankine}
Baseadas na escala Kelvin, e menos comuns. Relacionam-se através das equações:

\begin{align*}
    \degree R & = 1.8 \cdot K \\
    \degree F & = \degree R - 459.67 \\
    \degree F & = 1.8 \cdot \degree C + 32 
\end{align*}

\noindent A relação entre as 4 escalas nos pontos fixos é representada na tabela \ref{tab:temp-units}

\begin{table}[ht]
    \centering
    \begin{tabular}{lcccc}
        \hline
        & \unit{\kelvin} & \degree C & \degree R & \degree F \\
        \hline
        Zero Absoluto & 0.00 & -273.15 & 0.00 & -459.67 \\
        Ponto de fusão & 273.15 & 0.00 & 491.67 & 32.0 \\
        Ponto triplo & 273.16 & 0.01 & 491.69 & 32.02 \\
        Ponto de Ebulição & 373.15 & 100.0 & 671.67
 & 212 \\
        \hline
    \end{tabular}
    \caption{Relações entre escalas de temperatura}
    \label{tab:temp-units}
\end{table}