\section{Energia: Calor e Trabalho. 1ª lei da termodinâmica}

\subsection{Conhecimento prévio}

Já conhecemos uma série de definições relacionadas com conceitos energéticos:
\begin{gather}
    \label{eq:work}
    W_{\vec{F}} = \int_{s_1}^{s_2} \vec{F} \cdot d\vec{s} \\
    \label{eq:KE}
    \Delta KE = \frac{1}{2} m \left( V_2^2 - V_1^2 \right) \\
    \label{eq:PE}
    \Delta PE = -mg(z_2 - z_1) \\
    \label{eq:EM}
    \Delta E_m = \Delta KE + \Delta PE
\end{gather}
sendo (\ref{eq:work}) a definição de trabalho de uma força, (\ref{eq:KE}) a (variação de) energia cinética, (\ref{eq:PE}) a (variação de) energia potencial, e (\ref{eq:EM}) a (variação de) energia mecânica. A partir da equação \ref{eq:EM} deduz-se o princípio da conservação da energia.

São também já conhecidas as relações entre as grandezas:
\begin{gather*}
    \Delta KE = W_{\vec{F_R}} \\
    \Delta PE = W_{\vec{F_C}} \\
    \Delta E_m = W_{\vec{F_{NC}}}
\end{gather*}
sendo $\vec{F_R}$ a resultante das forças, $\vec{F_C}$ o somatório das forças conservativas, e $\vec{F_{NC}}$ o somatório das forças não conservativas.



\subsection{Trabalho}

Em termodinâmica uma interação é caracterizada como trabalho se o efeito dessa interação na vizinhança do sistema puder ser substituida pela subida ou descida de um peso.

Trabalho é um processo de transferência de energia.

\subsubsection{Convenção de sinais e notação}

\begin{center} \large
    Trabalho feito \textbf{\textit{pelo}} sistema: $ W>0 $
    
    Trabalho feito \textbf{\textit{no}} sistema: $ W<0 $
\end{center}


\paragraph{Diferencial do trabalho}

O trabalho não é uma propriedade do sistema. Ao contrário de outras grandezas, como o volume ou a pressão, o trabalho não tem um valor definido para cada estado, apenas sendo definido para um processo, ou seja, mudança de un estado para outro.

Assim sendo adota-se a notação $\delta W$  para o diferencial do trabalho em vez do expectável $\xcancel{dW}$.



\subsubsection{Potência}

A taxa de transferência de energia por trabalho é denominada por \textbf{Potência}:
\begin{equation*}
    \dot{W} = \frac{\delta W}{dt}
\end{equation*}

Quando o trabalho resulta de uma força macroscópica observável é válida a formula:
\begin{equation*}
    \dot{W} = \vec{F} \cdot \vec{V}
\end{equation*}



\subsubsection{Trabalho de expansão ou compressão de um fluído}

O trabalho realizado por um sistema que consiste num fluido dentro de um pistão quando este é deslocado é dado por:
\begin{equation*}
    \delta W = pA\,dx
\end{equation*}
sendo $p$ a pressão do fluído e $A$ a área da face do pistão.

O produto $A\,dx$ representa uma mudança de volume, $dV$. A equação pode ser integrada para uma mudança de volume de $V_1$ a $V_2$ obtendo-se a seguinte expressão, válida para todos os sistemas, em processos quase-estáticos:
\begin{equation} \label{eq:comp-exp-work}
    W = \int_{V_1}^{V_2} p\,dV
\end{equation}

\paragraph{Processos politrópicos}
Para avaliar o integral da equação (\ref{eq:comp-exp-work}) é necessário obter uma relação entre a pressão $p$ e o volume $V$. Uma das relações que pode ser considerada é aquela que representa um processo politrópico:
\begin{gather*}
    pV^{n} = \text{constante} \\
    \text{OU} \\
    pv^{n} = \text{constante}
\end{gather*}
Outras relações entre $p$ e $V$ podem ser consideradas.


\subsubsection{Outros exemplos de trabalho}

\paragraph{Extensão de uma barra sólida}
Numa barra fixa em $x=0$ à qual é aplicada uma tensão normal $\sigma$ o trabalho é dado por:
\begin{equation*}
    W = -\int_{x_1}^{x_2} \sigma A\,dx
\end{equation*}

\paragraph{Extensão de um filme de líquido}
Para um líquido suspenso numa estrutura de arame, sustentado pela tensão superficial do mesmo. Sendo $\tau$ a tensão superficial exercida pelo filme num arame móvel, o trabalho é dado por:
\begin{equation*}
    W = -\int_{A_1}^{A_2} \tau\,dA
\end{equation*}

\paragraph{Potência transmitida por um eixo}
Num eixo a rodar com velocidade angular $\omega$ que exerce um torque $\tau$ na vizinhança, o trabalho realizado em $n$ rotações é dado por:
\begin{equation*}
    W=2\pi n \tau
\end{equation*}
e a equação da potência transmitida para a vizinhanca é:
\begin{equation*}
    \dot{W} = \tau\omega
\end{equation*}

\paragraph{Trabalho elétrico}
Num sistema constituido por uma bateria de diferença de potencial $\varepsilon$, conectada a um circuito externo com uma corrente $I$. Sendo $Q$ a carga que atravessa um ponto do circuito, o trabalho é dado por
\begin{equation*}
    \delta W = -\varepsilon\,dQ
\end{equation*}
e a potência transmitida é
\begin{equation*}
    \dot{W} = -Ei
\end{equation*}



\subsection{Energia}

\noindent Em termodinâmica a energia é dividida em 3 componentes:
\begin{itemize}
    \item \textbf{Energia Cinética $KE$} - Associada com o movimento do sistema como um todo.
    \item \textbf{Energia Potencial Gravítica $PE$} - Relacionada com a posição do sistema como um todo, no campo gravítico terrestre.
    \item \textbf{Energia Interna $U$} - Todas as outras formas de energia.
\end{itemize}

\noindent Assim sendo, a energia total de um sistema é dada por:
\begin{equation*}
    E = KE + PE + U
\end{equation*}
e a variação da energia é representada por:
\begin{equation*}
    \Delta E = \Delta KE + \Delta PE + \Delta U
\end{equation*}



\subsection{Calor}
Até agora foram consideradas as transferências de energia que podem ser consideradas trabalho, mas é possivel haver outras formas de transferencia de enrgia. A este tipo de interação chama se \textbf{calor}.

\subsubsection{Convenção de sinais e notação}

Calor é a transferência de energia de e para um sistema. Por convenção considera se positiva quando o sistema ganha energia e negativa quando o sistema transfere energia para a vizinhança.

\begin{center} \large
    Energia transferida \textbf{\textit{para}} o sistema: $ Q>0 $
    
    Energia transferida \textbf{\textit{do}} sistema: $ Q<0 $
\end{center}

Tal como o trabalho, o calor não é uma propriedade do sistema, logo não tem um valor definido para um estado, sendo tambem usada a notação $\delta Q$ em vez de \xcancel{dQ}.

\subsubsection{Calor num processo}
O calor transferido num processo entre o estado 1 e o estado 2 é dado por:
\begin{equation*}
    Q = \int_{1}^{2} \delta Q
\end{equation*}

Para calcular o integral é útil definir uma grandeza denominada \textbf{taxa de transferência de calor} dada por:
\begin{equation*}
    \dot{Q} = \frac{\delta Q}{dt}
\end{equation*}
sendo assim a energia transferida por calor entre os instantes $t_1$ e $t_2$ dada por:
\begin{equation*}
    Q = \int_{t_1}^{t_2} \dot{Q}\,dt
\end{equation*}

\subsubsection{Modos de transferência de calor}
Existem 3 modos básicos de tranferencia de energia por calor: condução, convecção e radiação.



\subsection{Primeira Lei da termodinâmica - Balanço energético}

Das experiencia de Joule, entre outros, surgiu a primeira lei da termodinâmica:
\begin{center}
    \noindent\fbox{
        \parbox{0.8\textwidth}{             \centering\textit{
                A variação de \textbf{energia total} de um sistema fechado é igual à diferenca entre a energia transderida para o sistema sobre a forma de calor e o trabalho realizado pelo sistema sobre a vizinhança.
            }
        }
    }
\end{center}

Esta é expressa pela equação:
\begin{equation}
    \Delta U = Q - W
\end{equation}
em que o significado das variáveis está expresso na tabela \ref{tab:1-lei-variaveis}.

\begin{table}[ht]
    \centering
    \begin{tabular}{llc}
        \hline
        Símbolo & Variável & Unidades \\
        \hline
        $\Delta E$ & Variação de energia total & Joule (\unit{J}) \\
        $Q$ & Energia transferida para o sistema por calor & Joule (\unit{J}) \\
        $W$ & Energia transferida para a vizinhança por trabalho do sistema & Joule (\unit{J}) \\
        \hline

    \end{tabular}
    
    \caption{Variáveis na 1ª Lei da Termodinâmica}
    \label{tab:1-lei-variaveis}
\end{table}



\subsection{Análise de Ciclos}
Quando são analisados ciclos de um sistema, a variação de energia é nula.
\begin{gather*}
    \cancelto{0}{\Delta E} = Q - W \\
    \Leftrightarrow Q = W
\end{gather*}

Se considerarmos o calor de entrada $Q_{in}$ e o calor de saída $Q_{out}$ como quantidades positivas, o trabalho total é dado por:
\begin{equation*}
    W_{ciclo} = Q_{in} - Q_{out}
\end{equation*}

\subsubsection{Ciclos de Potência}
Um ciclo de potência é uma série de processos que converte energia térmica em trabalho. A eficiência térmica (rendimento) do ciclo é calculado com a equação:
\begin{equation}
    \eta = \frac{W_{ciclo}}{Q_{in}}
\end{equation}

\subsubsection{Ciclos de refrigeração e bombas de calor}
Um ciclo de refrigeração / bomba de calor é um sistema que transmite energia por calor de uma uma fonte para um acomulador, que se encontra a temperatura mais elevada, necessitando um trabalho positivo para tal.

A eficiência (coeficiente de performance) de um \textbf{Ciclo de Refrigeração} é dada por:
\begin{equation}
    \beta = \frac{Q_{in}}{W_{ciclo}}
\end{equation}
e para uma \textbf{Bomba de Calor} calcula se com:
\begin{equation}
    \gamma = \frac{Q_{out}}{W_{ciclo}}
\end{equation}

\subsubsection{Limitações de eficiência}
O rendimento ($\eta$) tem o limite teórico de 1, e a eficiência ($\beta$ e $\gamma$) não têm limite superior. No entanto, devido à 2ª Lei da Termodinâmica, que será estudada mais tarde, todos os coeficientes de performance, em aplicacoes reais, tem limites inferiores a esses valores teóricos.