\section{Introdução}

\subsection{Volume de controlo (pag. 16)}

Em Dinâmica de Fluidos vamos trabalhar principalmente com volumes de controlo, ou seja, sistemas abertos. Há fluxo de energia e massa entre o sistema e a vizinhança (praticamente) sempre. Assim sendo, torna-se quase impossível analisar um sistema aberto, usando téncicas Newtonianas ou Lagrangianas, pois estas envolvem obter a posição, velocidade, e aceleração de todas as partículas.

Utiliza-se assim o método Euleriano, onde as propriedades de um fluxo de fluido são dadas como funções das coordenadas espaciais, e do tempo. Abaixo encontra-se o exemplo da velocidade, descrita no método de Euler.

\subsection{Campo de velocidades (pag. 17)}
A velocidade é uma função vetorial da posição e do tempo, e as suas componentes são por convenção $u$, $v$ e $w$.

A velocidade é das propriedades mais fundamentais de um fluxo, pois muitas outras propriedades obtém-se através da velocidade.

\begin{equation*}
    \vec{V}(x,y,z,t) = u(x,y,z,t)\hat{i} + v(x,y,z,t)\hat{j} + w(x,y,z,t)\hat{k}
\end{equation*}

\subsection{Linhas de corrente / streamlines (pag. 41)}
Uma linha de corrente é uma curva tangente à velocidade em cada ponto, num dado instante.

\begin{equation*}
    \frac{dx}{u} = \frac{dy}{v} = \frac{dz}{w} = \frac{dr}{V}
\end{equation*}