\section{Relações integrais para um volume de controlo}

\subsection{Caudal e fluxo de massa (pag. 138)}
Qual o volume de fluido que atravessa uma superficie $S$ por unidade de tempo? Como $\vec{V}$ varia com a posição, devemos integrar em ordem a $dA$, e ter em conta que a velocidade pode fazer um ângulo $\theta$ com a normal à superficie. Assim sendo, o volume que atravessa a superfície é dado por:
\begin{equation*}
    d\mathcal{V} = V\,dt\,dA\cos{\theta} = (\vec{V}\cdot\hat{n})\,dA\,dt
\end{equation*}
sendo $\mathcal{V}$ o volume de fluido. (não confundir com $\vec{V}$ que representa a velocidade)

O integral de $d\mathcal{V}$ dá nos o \textbf{caudal} (fluxo de volume):
\begin{equation*}
    Q = \frac{d\mathcal{V}}{dt} = \int_S \vec{V}\cdot\hat{n}\,dA
\end{equation*}

Para obter o \textbf{fluxo de massa}, multiplica-se o fluxo de volume pela densidade:
\begin{equation} \label{eq:mass-flow-rate}
    \dot{m} = \frac{dm}{dt} = \int_S \rho\vec{V}\cdot\hat{n}\,dA
\end{equation}
Se a densididade e a velocidade forem uniformes na superfície, então obtém se a aproximação unidimensional:
\collectequation{DF-formulas}{Caudal e fluxo de massa (aprox. unidimensional)}{
    \dot{m} = \rho Q =\rho V A
}

Por convenção define-se $\hat{n}$ como o vetor normal que aponta para fora. Assim sendo, $Q$ e $\dot{m}$ indicam fluxo para fora quando são positivos, e para dentro quando são negativos.

As unidades de $Q$ são \unit{\meter\cubed\per\second} e de $\dot{m}$ são \unit{\kilogram\per\second}



\subsection{Teorema do transporte de Reynolds  (pag. 139)}
Seja $B$ uma propriedade qualquer do fluido e $\beta = \frac{dB}{dm}$ o seu valor específico, a quantidade total de $B$ no volume de controlo é dada por:
\begin{equation*}
    B_{CV} = \int_{CV}\beta\,dm = \int_{CV}\beta\rho\,d\mathcal{V}
\end{equation*}

Interessa-nos saber a taxa de variação temporal da propriedade $B$, e esta é dada pelo \textbf{Teorema do transporte de Reynolds}:
\begin{equation}
    \frac{dB_{Sist}}{dt} = \frac{d}{dt}\left( \int_{CV}\beta\rho\,d\mathcal{V} \right) + \int_{CS}\beta\rho(\vec{V}\cdot\hat{n})\,dA
\end{equation}
onde $\vec{V}$ é a \textbf{velocidade relativa} do fluido relativamente superfície do volume de controlo. CV é o volume de controlo e CS é a superfície de controlo.

Se o volume for fixo a derivada temporal passa para dentro do integral, obtendo-se:
\begin{equation*}
    \frac{dB_{Sist}}{dt} = \int_{CV}\frac{\partial}{\partial t}(\beta\rho)\,d\mathcal{V} + \int_{CS}\beta\rho(\vec{V}\cdot\hat{n})\,dA
\end{equation*}

Em muitos casos o volume é fixo, e tem um número definido de entradas e saidas unidimensionais - a velocidade é perpendicular a cada secção e as propriedades são uniformes em cada secção. Nestes casos, o teorema de transporte de Reynolds pode escrever-se como:

\collectequation{DF-formulas}{EQ. Transporte Reynolds}{
    \frac{dB_{Sist}}{dt} = \int_{CV}\frac{\partial}{\partial t}(\beta\rho)\,d\mathcal{V} + \left.\sum\beta_i \dot{m}_i\right|_{out} - \left.\sum\beta_i \dot{m}_i\right|_{in}
}

onde $\dot{m}_i = \rho_iA_iV_i$, obtendo-se assim:
\begin{equation*}
    \frac{dB_{Sist}}{dt} = \int_{CV}\frac{\partial}{\partial t}(\beta\rho)\,d\mathcal{V} + \sum(\rho_i\beta_iV_iA_i)_{out} - \sum(\rho_i\beta_iV_iA_i)_{in}
\end{equation*}



\subsection{Equação da conservação da massa (pag. 147)}
Até agora usamos $B$ para uma qualquer propriedade do sistema, e $\beta$ para o valor especifico (por unidade de massa) da propriedade.

Aplicando agora o teorema do transporte de Reynolds à massa $m$, onde a massa específica $\frac{dm}{dm} = 1$ obtém-se a \textbf{equação da conservação da massa}, com volume fixo:

\begin{equation*}
    \frac{dm_{Sist}}{dt} = 0 = \int_{CV}\frac{\partial\rho}{\partial t}\,d\mathcal{V} + \int_{CS}\rho(\vec{V}\cdot\hat{n})\,dA
\end{equation*}

No caso especifico de fluxo estável, a equação da conservação da massa reduz-se a:
\begin{equation*}
    \int_{CS}\rho(\vec{V}\cdot\hat{n})\,dA = 0
\end{equation*}
que, para um sistema com entradas e saídas unidimensionais, pode escrever-se como:
\begin{equation*}
    \sum(\rho_i A_i V_i)_{out} - \sum(\rho_i A_i V_i)_{in} = 0
\end{equation*}



\subsection{Equação do momento linear (pag. 152)}

\warnbox{AVISO: Notação}{
    Na próxima equação, é usada a notação $\vec{p}$ para o momento linear, que não deve ser confundida com a notação usada nos restantes capítulos onde $p$ (normalmente representada sem vetor) que representa a pressão.
}

Aplicando o teorema do transporte de Reynolds ao momento linear $\tcboxmath[colback=yellow!15, colframe=yellow!80!orange, top=0pt, bottom=0pt, left=0pt, right=0pt]{\vec{p} = m\vec{V}}$, onde o valor especifico é a velocidade $\vec{V}$, obtém-se a seguinte relação para o momento linear:
\begin{equation*}
    \frac{d\vec{p}_{Sist}}{dt} = \sum \vec{F} = \int_{CV}\frac{\partial(\rho\vec{V})}{\partial t}\,d\mathcal{V} + \int_{CS}\rho\vec{V}(\vec{V}\cdot\hat{n})\,dA
\end{equation*}
sendo esta válida para um volume de controlo fixo, com a velocidade observada num referencial inercial.

O termo $\sum \vec{F}$ representa a soma de todas as forças exercidas no fluído:
\begin{itemize}
    \item \textbf{Forças de superficie} \\
        Forças que atuam na fronteira do volume de controlo, como:
        \begin{itemize}
            \item Forças de pressão
            \item Forças viscosas (atrito)
            \item Forças exercidas por corpos sólidos (ex. pistão), sendo estas uma pressão distribuída na superfície de contacto.
        \end{itemize}
    \item \textbf{Forças volúmicas} \\
        Forças que atuam em todo o volume de controlo, como:
        \begin{itemize}
            \item Força da gravidade
            \item Forças eletromagnéticas
        \end{itemize}
\end{itemize}

\subsubsection{Força total da pressão}
As \textbf{forças de superficie} que atuam num volume de controlo são devidas \textbf{(1)} a forças \textbf{exercidas por corpos sólidos} que intersetam o volume de controlo (ex. pistão), e \textbf{(2)} a forças de \textbf{pressão} e \textbf{viscosidade} exercidas pelo \textbf{fluido envolvente} (quando a superficie de controlo interseta o fluido).

A força total da pressão é dada por:
\begin{equation} \label{eq:pressure-forces-abs}
    \vec{F}_{press} = \int_{CS}p(-\hat{n})\,dA
\end{equation}
Se a pressão for uniforme, $p=p_a$ em toda a superficie de controlo, então a força total de pressão é nula.
\begin{equation*}
    \vec{F}_{press} = -p_a\int_{CS}\hat{n}\,dA\ \tcboxmath[left=2pt, right=2pt]{= 0}
\end{equation*}

\errorbox{ATENÇÃO: vetor normal e sinais}{
    A convenção para o vetor normal $\hat{n}$ é que este aponta para fora do volume de controlo.
    
    A força de \textbf{pressão atua de fora para dentro} do volume de controlo, ou seja, no sentido oposto ao vetor normal, daí o sinal negativo. Este pode ser colocado no vetor normal, como em \eqref{eq:pressure-forces-abs} ou fora do integral, como em \eqref{eq:pressure-forces-gage}.
}

Muitos problemas podem ser simplificados se a pressão uniforme, $p_a$ for subtraída de todas as pressões envolvidas, sendo assim considerada, por definição desta a \textbf{pressão manométrica}. A equação da força total da pressão pode então ser escrita como:
\begin{equation} \label{eq:pressure-forces-gage}
    \vec{F}_{press} = -\int_{CS}(p-p_a)\hat{n}\,dA = -\int_{CS}p_{gauge}\hat{n}\,dA
\end{equation}
Um exemplo de pressão uniforme, $p_a$ é a pressão atmosférica, que atua em todas as superfícies expostas ao ar.

Para um volume de controlo com entradas e saídas unidimensionais, a força total da pressão pode ser escrita como:
\collectequation{DF-formulas}{Força total da pressão (unidimensional)}{
    F_{x\ press} = \sum p_{i} A_i (\hat{n}\cdot\vec{e}_x)
}

\warnbox{ATENÇÃO: Sinais}{
    O sinal positivo ou negativo depende da orientação do vetor normal e da direção positiva do eixo.

    \begin{equation*}
        (\hat{n}\cdot\vec{e}_x) = \begin{cases}
            +1 & \text{$\hat{n}$ no sentido do eixo} \\
            -1 & \text{$\hat{n}$ no sentido oposto ao eixo} \\
            0 & \text{$\hat{n}$ perpendicular ao eixo} \\
            \cos{\theta} & \text{caso geral}
        \end{cases}
    \end{equation*}

    A \textbf{pressão} atua sempre no sentido \textbf{de fora para dentro} do volume de controlo.
}



\subsection{Equação de Bernoulli (pag. 168)}
\begin{equation*}
    \boxed{
        \frac{p_1}{\rho} + \frac{1}{2}V_1^2 + gz_1 = \frac{p_2}{\rho} + \frac{1}{2}V_2^2 + gz_2 = \text{const}
    }
\end{equation*}
