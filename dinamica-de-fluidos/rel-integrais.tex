\section{Relações integrais para um volume de controlo}

\subsection{Caudal (pag. 138)}
Qual o volume de fluido que atravessa uma superficie $S$ por unidade de tempo? Como $\vec{V}$ varia com a posição, devemos integrar em ordem a $dA$, e ter em conta que a velocidade pode fazer um ângulo $\theta$ com a normal à superficie. Assim sendo, o volume que atravessa a superfície é dado por:
\begin{equation*}
    d\mathcal{V} = V\,dt\,dA\cos{\theta} = (\vec{V}\cdot\hat{n})\,dA\,dt
\end{equation*}
sendo $\mathcal{V}$ o volume de fluido. (não confundir com $\vec{V}$ que representa a velocidade)

O integral de $d\mathcal{V}$ dá nos o \textbf{caudal} (fluxo de volume):
\begin{equation*}
    Q = \frac{d\mathcal{V}}{dt} = \int_S \vec{V}\cdot\hat{n}\,dA
\end{equation*}

Para obter o fluxo de massa, multiplica-se o fluxo de volume pela densidade:
\begin{equation*}
    \dot{m} = \frac{dm}{dt} = \int_S \rho\vec{V}\cdot\hat{n}\,dA
\end{equation*}
Por convenção define-se $\vec{n}$ como o vetor normal que aponta para fora. Assim sendo, $Q$ e $\dot{m}$ indicam fluxo para fora quando são positivos, e para dentro quando são negativos.

As unidades de $Q$ são \unit{\meter\cubed\per\second} e de $\dot{m}$ são \unit{\kilogram\per\second}



\subsection{Teorema do transporte de Reynolds}
Seja $B$ uma propriedade qualquer do fluido e $\beta = \frac{dB}{dm}$ o seu valor específico, a quantidade total de $B$ no volume de controlo é dada por:
\begin{equation*}
    B_{CV} = \int_{CV}\beta\,dm = \int_{CV}\beta\rho\,d\mathcal{V}
\end{equation*}

Interessa-nos saber a taxa de variação temporal da propriedade $B$, e esta é dada pelo \textbf{Teorema do transporte de Reynolds}:
\begin{equation}
    \boxed{
        \frac{dB_{Sist}}{dt} = \frac{d}{dt}\left( \int_{CV}\beta\rho\,d\mathcal{V} \right) + \int_{CS}\beta\rho(\vec{V}\cdot\hat{n})\,dA
    }
\end{equation}
onde $\vec{V}$ é a \textbf{velocidade relativa} do fluido relativamenet superfície do volume de controlo.

Se o volume for fixo a derivada temporal passa para dentro do integral, obtendo-se:
\begin{equation*}
    \frac{dB_{Sist}}{dt} = \int_{CV}\frac{\partial}{\partial t}(\beta\rho)\,d\mathcal{V} + \int_{CS}\beta\rho(\vec{V}\cdot\hat{n})\,dA
\end{equation*}

Em muitos casos o volume é fixo, e tem um número definido de entradas e saidas unidimensionais - a velocidade é perpendicular a cada secção e as propriedades são uniformes em cada secção. Nestes casos, o teorema de transporte de Reynolds pode escrever-se como:
\begin{equation}
    \boxed{
        \frac{dB_{Sist}}{dt} = \int_{CV}\frac{\partial}{\partial t}(\beta\rho)\,d\mathcal{V} + \left.\sum\beta_i \dot{m}_i\right|_{out} - \left.\sum\beta_i \dot{m}_i\right|_{in}
    }
\end{equation}
onde $\dot{m}_i = \rho_iA_iV_i$, obtendo-se assim:
\begin{equation*}
    \frac{dB_{Sist}}{dt} = \int_{CV}\frac{\partial}{\partial t}(\beta\rho)\,d\mathcal{V} + \sum(\rho_i\beta_iV_iA_i)_{out} - \sum(\rho_i\beta_iV_iA_i)_{in}
\end{equation*}

\subsubsection{Teorema do transporte de Reynolds aplicado à massa}
Até agora usamos $B$ para uma qualquer propriedade do sistema, e $\beta$ para o valor especifico (por unidade de massa) da propriedade.

Aplicando agora o teorema do transporte de Reynolds à massa $m$, onde a massa específica $\frac{dm}{dm} = 1$ obtém-se a \textbf{equação da conservação da massa}