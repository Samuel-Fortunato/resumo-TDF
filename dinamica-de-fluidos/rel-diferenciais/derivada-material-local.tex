\section{Relações diferenciais para um escoamento}
Com o teorema do transporte de Reynolds podemos aplicar leis da Mecanica e da Termodinâmica a volumes de controlo finitos. Pretendemos agora aplicar as mesmas leis a partículas materiais de um meio contínuo - isto é, quantidades de massa muito pequenas.

\subsection{Derivada material e derivada local}
Para aplicar leis da mecanica a partículas materiais temos que relacionar a derivada local de uma propriedade, com a derivada segundo o movimento das partículas.

Considerando uma propriedade com função da posição e do tempo, a \textbf{derivada material} (derivada de uma propriedade de uma partícula de matéria que se move com velocidade $\vec{v}$) é dada por:
\collectequation{DF-formulas}{Derivada Material}{
        Derivada de uma propriedade de uma partícula de matéria que se move com velocidade $\vec{v}$
    }{
        \frac{D}{Dt} = \frac{\partial}{\partial t} + \left(\vec{v}(\vec{r},t)\cdot\vec{\nabla}\right) = \frac{\partial}{\partial t} + \left(u\frac{\partial }{\partial x} + v\frac{\partial }{\partial y} + w\frac{\partial }{\partial z}\right)
}

O primeiro termo denomina-se termo local ou \textbf{derivada local}. O segundo chama-se \textbf{termo convectivo}

\warnbox[0.6\textwidth]{Notação de derivada material}{
    Por convenção usa se o $D$ maiúsculo para indicar derivada material, e para não confundir com a derivada local.

    \begin{equation*}
        \frac{D}{Dt} \equiv \frac{d}{dt} \neq \frac{\partial}{\partial t} 
    \end{equation*}

    Daqui em diante usaremos apenas a notação com $D$ maiúsculo
}




