\subsection{Equação do momento linear (Navier-Stokes)}
A equação para o momento linear é dada por:
\begin{equation*}
    \sum\vec{F} = \rho \frac{D\vec{V}}{Dt}\,dx\,dy\,dz
\end{equation*}

As forças que atuam na partícula são as forças volúmicas (apenas consideraremos a gravidade) e as forças de contacto. Estas últimas são devidas às tensões na superficie do volume de controlo (infinitesimal), que são a soma da pressão hidrostática, $p$ com as tensões viscosas $\tau_{ij}$.

Substituindo as forças na equação do momento linear obtemos a seguinte equação, na forma vetorial:
\collectequation{DF-formulas}{Equação do momento linear}{Forma vetorial}{
    \rho\frac{D\vec{V}}{Dt} = \rho\vec{g} - \vec{\nabla}p + \vec{\nabla}\cdot\overline{\overline{\tau}}
}
onde
\begin{equation*}
    \overline{\overline{\tau}} = \begin{bmatrix}
        \tau_{xx} & \tau_{yx} & \tau_{zx} \\
        \tau_{xy} & \tau_{yy} & \tau_{zy} \\
        \tau_{xz} & \tau_{yz} & \tau_{zz} \\
    \end{bmatrix}
\end{equation*}

Esta equação está escrita de uma forma extremamente compacta, mas se for escrita por extenso, para cada componente, toma a forma:

\collectequation{DF-formulas}{Equação do momento linear para a componente $x$}{Equivalente para as componentes $y$ e $z$}{
    \rho\frac{Du}{Dt} = \rho g_x - \frac{\partial p}{\partial x} + \frac{\partial \tau_{xx}}{\partial x} + \frac{\partial \tau_{yx}}{\partial y} + \frac{\partial \tau_{zx}}{\partial z}
}



\subsubsection{Fluidos não viscosos - Equação de Euler}
Caso a viscosidade seja desprezável (fluidos ideiais) assumimos que $\overline{\overline{\tau}} = 0$, a equação do momento linear reduz-se a:
\begin{equation*}
    \rho\frac{D\vec{V}}{Dt} = \rho\vec{g} - \vec{\nabla}p
\end{equation*}



\subsubsection{Fluidos Newtonianos - Equação de Navier-Stokes}
Para \textbf{fluidos newtonianos} as tensões viscosas são proporcionais à velocidade e ao coeficiente de viscosidade. Para escoamento incompressível tem-se que:
\begin{gather*}
    \tau_{xx} = 2\mu \frac{\partial u}{\partial x}, \qquad
    \tau_{yy} = 2\mu \frac{\partial v}{\partial y}, \qquad
    \tau_{zz} = 2\mu \frac{\partial w}{\partial z} \\[1em]
    \tau_{xy} = \tau_{yx} = \mu \left( \frac{\partial u}{\partial y}
    + \frac{\partial v}{\partial x} \right) \\
    \tau_{xz} = \tau_{zx} = \mu \left( \frac{\partial w}{\partial x}
    + \frac{\partial u}{\partial z} \right) \\
    \tau_{yz} = \tau_{zy} = \mu \left( \frac{\partial v}{\partial z}
    + \frac{\partial w}{\partial y} \right)
\end{gather*}
onde $\mu$ é o coeficiente de viscosidade.

Substituindo na equação do momento linear obtém-se as \textbf{equações de Navier-Stokes}, expressas na forma vetorial:
\collectequation{DF-formulas}{Equação de Navier-Stokes}{
    Escoamento incompressível, fluido Newtoniano, forma vetorial
}{
    \rho\frac{D\vec{V}}{Dt} = \rho\vec{g} - \vec{\nabla}p + \mu\vec{\nabla}^2\vec{V}
}

Ou expressas forma extensa, para a componente em $x$:
\collectequation{DF-formulas}{Equação de Navier-Stokes}{
    Mesmas condições que o anterior, componente em x
}{
    \rho\frac{Du}{Dt} = \rho g_x - \frac{\partial p}{\partial x} + \mu\left(\frac{\partial^2 u}{\partial x^2} + \frac{\partial^2 u}{\partial y^2} + \frac{\partial^2 u}{\partial z^2}\right)
}
