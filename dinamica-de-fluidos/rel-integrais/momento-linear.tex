\subsection{Equação do momento linear (pag. 152)}

\warnbox{AVISO: Notação}{
    Na próxima equação, é usada a notação $\vec{p}$ para o momento linear, que não deve ser confundida com a notação usada nos restantes capítulos onde $p$ (normalmente representada sem vetor) que representa a pressão.
}

Usando a 3ª lei de Newton: $\mathwarn{\vec{F}_R = \dot{\vec{p}}}$ e aplicando o teorema do transporte de Reynolds ao momento linear $\mathwarn{\vec{p} = m\vec{V}}$, onde o valor especifico é a velocidade ($\beta = d\vec{p}/dm = \vec{V}$), obtém-se a seguinte relação para o momento linear:
\begin{equation*}
    \sum \vec{F} = \frac{d\vec{p}_{Sist}}{dt} = \int_{CV}\frac{\partial(\rho\vec{V})}{\partial t}\,d\mathcal{V} + \int_{CS}\rho\vec{V}(\vec{V}\cdot\hat{n})\,dA
\end{equation*}
sendo esta válida para um volume de controlo fixo, com a velocidade observada num referencial inercial.

Para um sistema com entradas e saídas unidimensionais, presente na maioria dos exercícios, usa se a versão simplificada:
\collectequation{DF-formulas}
    {Equação do momento linear}{
        Escoamento estável e incompressível, entradas e saídas unidimensionais.
    }{
        \sum \vec{F} = \int_{CV}\cancelto{0}{\frac{\partial\vec{V}}{\partial t}}\rho\,d\mathcal{V} + \left.\sum \vec{V}_i \dot{m}_i\right|_{out} - \left.\sum \vec{V}_i \dot{m}_i\right|_{in}
}


O termo $\sum \vec{F}$ representa a soma de todas as forças exercidas no fluído:
\begin{itemize}
    \item \textbf{Forças de superficie} \\
        Forças que atuam na fronteira do volume de controlo, como:
        \begin{itemize}
            \item Forças de pressão
            \item Forças viscosas (atrito)
            \item Forças exercidas por corpos sólidos (ex. pistão), sendo estas uma pressão distribuída na superfície de contacto.
        \end{itemize}
    \item \textbf{Forças volúmicas} \\
        Forças que atuam em todo o volume de controlo, como:
        \begin{itemize}
            \item Força da gravidade
            \item Forças eletromagnéticas
        \end{itemize}
\end{itemize}

\subsubsection{Força total da pressão (pag. 153)}
As \textbf{forças de superficie} que atuam num volume de controlo são devidas \textbf{(1)} a forças \textbf{exercidas por corpos sólidos} que intersetam o volume de controlo (ex. pistão), e \textbf{(2)} a forças de \textbf{pressão} e \textbf{viscosidade} exercidas pelo \textbf{fluido envolvente} (quando a superficie de controlo interseta o fluido).

A força total da pressão é dada por:
\begin{equation} \label{eq:pressure-forces-abs}
    \vec{F}_{press} = \int_{CS}p(-\hat{n})\,dA
\end{equation}
Se a pressão for uniforme, $p=p_a$ em toda a superficie de controlo, então a força total de pressão é nula.
\begin{equation*}
    \vec{F}_{press} = -p_a\int_{CS}\hat{n}\,dA\ \tcboxmath[left=2pt, right=2pt]{= 0}
\end{equation*}

\errorbox{ATENÇÃO: Vetor normal e sinais}{
    A convenção para o vetor normal $\hat{n}$ é que este aponta para fora do volume de controlo.
    
    A força de \textbf{pressão atua de fora para dentro} do volume de controlo, ou seja, no sentido oposto ao vetor normal, daí o sinal negativo. Este pode ser colocado no vetor normal, como em \eqref{eq:pressure-forces-abs} ou fora do integral, como em \eqref{eq:pressure-forces-gage}.
}

Muitos problemas podem ser simplificados se a pressão uniforme, $p_a$ for subtraída de todas as pressões envolvidas, sendo assim considerada, por definição desta a \textbf{pressão manométrica}. A equação da força total da pressão pode então ser escrita como:
\begin{equation} \label{eq:pressure-forces-gage}
    \vec{F}_{press} = -\int_{CS}(p-p_a)\hat{n}\,dA = -\int_{CS}p_{gauge}\hat{n}\,dA
\end{equation}
Um exemplo de pressão uniforme, $p_a$ é a pressão atmosférica, que atua em todas as superfícies expostas ao ar.

Para um volume de controlo com entradas e saídas unidimensionais, a força total da pressão pode ser escrita como:
\collectequation{DF-formulas}
    {Força total da pressão}
    {Entradas e saídas unidimensionais}{
        F_{\text{x press}} = \sum p_{i} A_i (-\hat{n}\cdot\vec{e}_x)
}

\warnbox{ATENÇÃO: Sinais}{
    O sinal positivo ou negativo depende da orientação do vetor normal e da direção positiva do eixo.

    \begin{equation*}
        (-\hat{n}\cdot\vec{e}_x) = \begin{cases}
            -1 & \text{$\hat{n}$ no sentido do eixo} \\
            1 & \text{$\hat{n}$ no sentido oposto ao eixo} \\
            0 & \text{$\hat{n}$ perpendicular ao eixo} \\
            -\cos{\theta} & \text{caso geral}
        \end{cases}
    \end{equation*}

    A \textbf{pressão} atua sempre no sentido \textbf{de fora para dentro} do volume de controlo.
}
