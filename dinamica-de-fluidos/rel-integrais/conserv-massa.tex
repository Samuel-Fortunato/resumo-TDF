\subsection{Equação da conservação da massa (pag. 147)}
\label{sec:conserv-massa-reynolds}

Até agora usamos $B$ para uma qualquer propriedade do sistema, e $\beta$ para o valor especifico (por unidade de massa) da propriedade.

Aplicando agora o teorema do transporte de Reynolds à massa $m$, onde a massa específica $\frac{dm}{dm} = 1$ obtém-se a \textbf{equação da conservação da massa}, com volume fixo:

\begin{equation*}
    \frac{dm_{Sist}}{dt} = 0 = \int_{CV}\frac{\partial\rho}{\partial t}\,d\mathcal{V} + \int_{CS}\rho(\vec{V}\cdot\hat{n})\,dA
\end{equation*}

No caso especifico de escoamento estável, a equação da conservação da massa reduz-se a:
\begin{equation*}
    \int_{CS}\rho(\vec{V}\cdot\hat{n})\,dA = 0
\end{equation*}
que, para um sistema com entradas e saídas unidimensionais, pode escrever-se como:
\begin{equation*}
    \sum(\rho_i A_i V_i)_{out} - \sum(\rho_i A_i V_i)_{in} = 0
\end{equation*}
